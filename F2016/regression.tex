\documentclass[12pt]{beamer}
\usetheme{Penn} 
\usepackage{amsmath, amssymb, amsthm, amsfonts}
\usepackage{graphicx}
\usepackage{fancyvrb}
\usepackage{verbatim}
%\usepackage{tabularx}
\usepackage{tikz}
\usepackage{animate}
\usetikzlibrary{matrix, shapes, arrows, calc, backgrounds}
\usepackage[vcentermath, enableskew]{youngtab}
\newcommand{\ZZ}{\ensuremath{\mathbb{Z}}}
\newcommand{\RR}{\ensuremath{\mathbb{R}}}
\newcommand{\PP}{\ensuremath{\mathbb{P}}}

\newcommand{\CC}{\ensuremath{\mathbb{C}}}

\newtheorem{thm}{Theorem}[section]
\newtheorem{cor}[thm]{Corollary}
\newtheorem{lem}[thm]{Lemma}
\newtheorem{prop}[thm]{Proposition}
\theoremstyle{definition}
\newtheorem{conj}[thm]{Conjecture}
\newtheorem{defn}[thm]{Definition}
\newtheorem{ex}[thm]{Example}
\newtheorem{rmk}[thm]{Remark}
\newtheorem{alg}[thm]{Algorithm}
\newtheorem{question}[thm]{Question}
\begin{document}

\author[Z. Rosen]{Zvi Rosen \\ Department of Mathematics}

\date[\today]{\today}
\title[Statistics \& Linear Regression]{{\Large Statistics \&
 Linear Regression}}
\institute[Dept. of Mathematics~~--~~University of Pennsylvania]{}


\frame{\titlepage}

\begin{frame}
\frametitle{Plan}
How do we analyze a set of data using MATLAB? \\[3mm]
In particular, how do we obtain representative statistics
for a set of real-number data? \\[3mm]
When our data points consist of two real-numbers, how
do we analyze the data set, and the relationship between
the two coordinates? \\[3mm]


\end{frame}


\begin{frame}
\frametitle{Mean, Median, Mode}
Let $X$ be a data set, equivalently, a point in $\RR^n$. \begin{itemize}

\item mean$(X)$ is the average of all values: $\frac{1}{n}\sum_{i = 1}^n x_i$.

\item median$(X)$, when $n$ is odd, is the value of index $(n-1)/2$ when
the data are ordered from smallest to largest. When $n$ is even,
it is the mean of the values with indices $n/2$ and $n/2 + 1$.

\item mode$(X)$ is the most commonly occurring value in the dataset. When
there is a tie, MATLAB gives the tie to the smallest value.
\end{itemize}

The MATLAB commands are {\tt mean, median}, and {\tt mode}.
\end{frame}

\begin{frame}
\frametitle{Representative Statistics from Metrics}
Suppose we use $d(p,q)$ to denote the distance between
points $p$ and $q$. Let $X$ be our data set, and let $Y$ be some
point with the same number in each coordinate 
(i.e. $(y,y,\ldots,y)$).

\begin{enumerate}
\item $d(X,Y) = \sum {\bf 1}[X_i \neq Y_i]$. (Hamming)
\item $d(X,Y) = \sum |X_i - Y_i|$. (Taxicab)
\item $d(X,Y) = \max |X_i - Y_i|$. ($\infty$-norm)
\item $d(X,Y) = \sqrt{\sum (X_i - Y_i)^2}$ ($2$-norm) 
\end{enumerate}
What value of $y$ minimizes the distance between
$X$ and $Y$?
\end{frame}

\begin{frame}
\frametitle{Standard Deviation \& Variance}
Mean, median, and mode give us a single number to 
estimate the whole data set. We often want 
an estimate for how spread out the values are
from that single estimate.

\begin{itemize}
\item Range: $\max(X) - \min(X)$.
\item Variance: $S_t = \sum (X_i - \bar{X})^2$, where $\bar{X}$ is the mean.
\item Standard Deviation: $s_y = \sqrt{S_t/(n-1)}$.
\end{itemize}
\end{frame}

\begin{frame}
\frametitle{Normal Distribution}
Many data distributions are distributed normally.
The normal distribution is defined as
\[ 
f(x) = -\frac{1}{\sqrt{2\pi} \sigma} \exp\left(-\dfrac{(x-\mu)^2}{2\sigma^2}\right)
\]
This is the well-known bell curve with mean $\mu$ and standard
deviation $\sigma$. The probability of any data point falling between
values $a$ and $b$ is equal to the integral of $f$ from $a$ to $b$.
\end{frame}

\begin{frame}
\frametitle{Generating random numbers in MATLAB}
To generate random numbers in the uniform distribution, i.e.
any range between $0$ and $1$ is equally likely, use the command
{\tt rand(m,n)}. This returns an $m \times n$ matrix of random
numbers. \\[3mm]

To generate random numbers in the normal distribution with
mean $0$ and standard deviation $1$, use the command
{\tt randn(m,n)}. This returns an $m \times n$ matrix of random
normal numbers. \\[3mm]

\begin{enumerate}
\item How do we generate random numbers in the uniform
distribution between $0$ and $100$?
\item How do we generate random numbers in the normal distribution
with mean $1$ and standard deviation $0.5$?
\end{enumerate}
\end{frame}

\begin{frame}
\frametitle{Regression}
Now, we consider data points with data points with
two coordinates $(x,y)$. We want to assess the hypothesis
that these two variables are related. In particular, can
we produce a line that is close to predicting their relationship? \\[4mm]

History: the term ''regression'' comes from a biological
phenomenon of children of tall parents regressing to 
average population height. It was eventually adopted to describe the
analysis of relationships between two data sets.
\end{frame}

\begin{frame}
\frametitle{Least-squares Error}

In order to find a line ``close'' to the data, we need to know
what ``close'' means. (pick your metric!) \\[5mm]

The taxicab metric does not specify a unique line. \\
The $\infty$-metric gives too much weight to outliers.\\

The Euclidean metric (corresponding to the 2-norm) is our best
option. In particular, we take the set of $x$-data points, and
compare the corresponding line values to the actual $y$-values.
This tells us how ``close'' to the data our line is.
\end{frame}

\begin{frame}
\frametitle{Least-squares Line}

To find the least-squares error, i.e. minimum Euclidean
distance, we use our optimization tool box.

Let $S_r$ be the sum of the residuals,\\
and let $a_0 + a_1\cdot x$ be the linear estimator. \\[5mm]

We take $dS_r/da_0$ and $dS_r/da_1$ to find the optimal
values of these parameters for the least-squares line.
\end{frame}

\begin{frame}
\frametitle{Quantifying Error}
Just like we measured spread from a
central estimator in one coordinate, we want to measure
spread from our regression line in two coordinates.

Our approach is to use the residuals:
\[ s_{y/x} = \sqrt{S_r}{n-2}.\]
(standard error of the estimate)

The goodness of the fit by our line can be described by:
\[ r^2 = \dfrac{S_t - S_r}{S_t}\]
($r$ is the correlation coefficient. 
This describes the proportion of the total variance in $y$-values
that can be described by the line as opposed to residual
effects)
\end{frame}

\end{document}
