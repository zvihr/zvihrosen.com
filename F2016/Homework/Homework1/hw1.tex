\documentclass[12pt]{amsart}
\usepackage[fullpage]{geometry}
\usepackage{fullpage}
\usepackage{pbox}
\usepackage{graphicx}
\usepackage{booktabs} % Top and bottom rules for table
\usepackage{amsfonts, amsmath, amsthm, amssymb}
\usepackage{longtable,array,color,xcolor}
\usepackage[colorlinks = true,
            urlcolor  = blue]{hyperref}
\usepackage{verbatim}
\usepackage{enumerate}
\newcommand\narrowstyle{\SetTracking{encoding=*}{-50}\lsstyle}

\setlength{\parindent}{0pt}

\begin{document}

\title{Math 320: Homework 1}
Due: September 9, 2016
\maketitle

Please read through chapters 2 and 3 of the textbook.
Obtain access to a copy of MATLAB or GNU Octave to answer the
following questions. For all questions, please write a brief
description of how your code works.

\begin{enumerate}
\item Problem 2.9: The density of freshwater can be computed
as a function of temperature with the following cubic equation:
\[\rho = 5.5289 \times 10^{-8} T_C^3 - 8.5016 \times 10^{-6} T_C^2  +6.5622 \times 10^{-5} T_C + 0.99987\]
where $\rho =$ density (g/cm$^3$) and $T_C = $ temperature ($^\circ C$).
Use MATLAB to generate a vector of temperatures ranging from $32^\circ F$ to
$93.2 ^\circ F$ using increments of $3.6^\circ F$. Convert this vector to degrees Celsius
and then compute a vector of densities based on the cubic formula. Create a plot of
$\rho$ versus $T_C$. Recall that $T_C = 5/9 (T_F - 32)$.

\vspace{1cm}
\item Problem 2.15: The Maclaurin series expansion for the cosine is
\[ \cos x = 1 - \frac{x^2}{2!} + \frac{x^4}{4!} - \frac{x^6}{6!} + \frac{x^8}{8!} - \cdots \]
Use MATLAB to create a plot of the cosine (solid line) along with a 
plot of the series expansion (black dashed line) up to and including the term
$x^8/8!$. Use the built-in function {\tt factorial} in computing the series 
expansion. Make the range of the abscissa from $x = 0 $ to $3 \pi /2$.

\vspace{1cm}

\item Problem 3.6: Two distances are required to specify the location of a point
relative to an origin in two-dimensional space:
\begin{itemize}
\item The horizontal and vertical distances $(x,y)$ in Cartesian coordinates; or
\item The radius and angle $(r,\theta)$ in polar coordinates.
\end{itemize}
It is relatively straightforward to compute Cartesian coordinates $(x,y)$ on the 
basis of polar coordinates $(r,\theta)$. The reverse process is not so simple.
The radius can be computed by the following formula: $r = \sqrt{x^2 + y^2}$.

If the coordinates lie within the first and fourth quadrants (i.e., $x > 0$), then
a simple formula can be used to compute $\theta$: $\theta = \tan^{-1}(y/x)$. The 
difficulty arises for the other cases. The following table summarizes the possibilities:
\[\begin{array}{l|l|l}
x & y & \theta \\ \hline
<0 & > 0 &  \tan^{-1}(y/x) + \pi \\
<0 & < 0 &   \tan^{-1}(y/x) - \pi \\
<0 & = 0 &  \pi \\
=0 & > 0 & \pi/2 \\
=0 & < 0 & -\pi/2 \\
=0 & = 0 & 0
\end{array}\]

Write a well-structured M-file using {\tt if...elseif} structures
to calculate $r$ and $\theta$ as a function of $x$ and $y$. Express
the final results for $\theta$ in degrees. Test your program by
evaluating the following cases:

\[\begin{array}{l|l|p{1cm}p{1cm}}
x & y & r & \theta \\ \hline
2 & 0 &  & \\
2 & 1 &  &  \\
0 & 3 &  & \\
-3 & 1 & & \\
-2 & 0 & & \\
-1 & -2 & & \\
0 & 0 & \\
0 & -2 & \\
2 & 2 & \\
\end{array}\]

\vspace{1cm}

\item Problem 3.20: A Cartesian vector can be thought of as representing
magnitudes along the $x-$, $y-$, and $z-$ axes multiplied by a unit vector $(i,j,k)$.
For such cases, the dot product of two of these vectors $\{a\}$ and $\{b\}$
corresponds to the product of their magnitudes and the cosine of the angle between
their tails as in $\{a \} \cdot \{b\} = ab \cos \theta$.

The cross product yields another vector, $\{c\} = \{a \} \times \{ b\}$, which
is perpendicular to the plane defined by $\{a\} $ and $\{ b\}$ such that its
direction is specified by the right-hand rule. Develop an M-file function
that is passed two such vectors and returns $\theta$, $\{c\}$ and the magnitude
of $\{c\}$ and generates a three-dimensional plot of the three vectors $\{a\},\{b\}$
and $\{c\}$ with their origins at zero. Use dashed lines for $\{a\}$ and $\{b\}$
and a solid line for $\{c\}$. Test your function for the following cases:

\begin{enumerate}[(a)]
\item \begin{verbatim} a = [6 4 2]; b = [2 6 4];\end{verbatim}
\item \begin{verbatim} a = [3 2 -6]; b = [4 -3 1];\end{verbatim}
\item \begin{verbatim} a = [2 -2 1]; b = [4 2 -4];\end{verbatim}
\item \begin{verbatim} a = [-1 0 0]; b = [0 -1 0];\end{verbatim}
\end{enumerate}

\end{enumerate}
\begin{comment}

\item Let $n$ be a positive integer.
We would like to list all strictly increasing sequences of 
positive integers with four terms: $a_1 < a_2 < a_3 < a_4$, such
that $a_1 + a_2 + a_3 + a_4 = n$. Please write an m-file defining
the function {\tt Sumfour(n)} which takes as input a positive integer,
and prints to the screen one partition per line.

For example, 
\begin{verbatim}
>> Sumfour(13)
1 + 2 + 3 + 7 
1 + 2 + 4 + 6 
1 + 3 + 4 + 5
\end{verbatim}
Please submit your code, and the output of {\tt Sumfour(17)}.
Hint: Your code might use a nested {\tt for} loop 
and the {\tt fprintf} command.

The Fibonacci sequence is defined as the sequence whose
first two terms are $s_1 = s_2 = 1$ and whose $n$-th
term  $s_n = s_{n-1} + s_{n-2'}$.

Write a function {\tt Fibonacci} that takes as input
a positive integer $n$ and prints every term 
of the Fibonacci sequence until the last term before it exceeds $n$.
\end{comment}


\end{document}
