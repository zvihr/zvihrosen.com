\documentclass[12pt]{amsart}
\usepackage[margin=1in]{geometry}
\usepackage{pbox}
\usepackage{graphicx}
\usepackage{booktabs} % Top and bottom rules for table
\usepackage{amsfonts, amsmath, amsthm, amssymb}
\usepackage{longtable,array,color,xcolor}
\usepackage[colorlinks = true,
            urlcolor  = blue]{hyperref}

\newcommand\narrowstyle{\SetTracking{encoding=*}{-50}\lsstyle}

\pagestyle{empty}


\setlength{\parindent}{0pt}

\begin{document}
{\bf Math 320 -- Computer Methods in the Mathematical Sciences I.} Fall 2016.\\[3mm]
Instructor: Zvi Rosen. \\
Office: DRLB 3N8D.\\
Website: Canvas.\\ %\url{math.upenn.edu/\~ zvihr/math320-F16.html}\\
Class Location: DRLB 3C2 \\
Class Times: MWF 11am-12pm \\
Office Hours: Wednesday 12pm-1pm or by appointment, DRLB 3N8D

\subsection*{Prerequisites} MATH 240. If you have not taken MATH 240,
but you would like to take the course,
please speak to the instructor. Programming experience may be useful, but
is not necessary to take the course.

\subsection*{Attendance}
Attendance is highly recommended but not mandatory. If you anticipate
missing multiple days of class, please let the instructor know in
advance.

\subsection*{Grading}
40 \% -- Homework \\
30 \% -- Quizzes \\
30 \% -- Final Project \\
If you would like to submit an assignment for regrading, you must
do so within a week of getting the grade.

\subsection*{Homework}
Homework will be assigned on a near-weekly basis. Teamwork is
encouraged, as long as you specify your colleagues in your writeup.
Each team member should write up and submit their work separately.
The lowest homework grade will be dropped. Homework will be
collected at the end of class on the day it is due.

\subsection*{Quizzes}
Quizzes will be administered most weeks on Friday. They will
cover the recent material studied in class.  Each quiz
will last about 20 minutes, and the lowest two quiz grades will be dropped.

\subsection*{Course Text}
The course textbook is \emph{Applied Numerical Methods with MATLAB for 
Engineers and Scientists, Third Edition} by Steven C. Chapra. 
Some homework assignments
will include problems from the textbook, so please get a copy!

\subsection*{Final Project}
The final project will entail a written proposal, a
full-length written report, and a 20 minute presentation 
at the end of the semester. The projects will
be carried out in pairs or individually. 
More details will come in a
separate handout.

\subsection*{Course Objectives}
We will begin the semester by becoming familiar with MATLAB.
Then we will delve into the following topics:
\begin{enumerate}
\item Roots and optimization.
\item Linear algebra.
\item Regression and fit.
\item Interpolation.
\item Differentiation and integration.
\item Differential equations.
\end{enumerate}

In each topic, we will discuss the theory underlying
computational methods, and then implement these techniques
in MATLAB. Our focus will be on proving convergence and
error bounds for our techniques.


\subsection*{Student Disabilities Services}
University of Pennsylvania, provides reasonable
 accommodations to students with disabilities who have 
self-identified and been approved by the office of
 Student Disabilities Services (SDS).  
Please make an appointment to meet with me 
as soon as possible in order to discuss your accommodations and your needs.

If you have not yet contacted SDS, and would 
like to request accommodations or have questions, 
you can make an appointment by calling SDS 215.573.9235.  
The office is located in the Weingarten Learning Resources 
Center at Stouffer Commons 3702 Spruce Street, Suite 300. 
All services are confidential.

\subsection*{Academic Integrity}
You can consult the University's Code of Academic Integrity here:
{\small \url{http://provost.upenn.edu/policies/pennbook/2013/02/13/code-of-academic-integrity}}.\\
These standards will be upheld in this class. If you use fragments of code from
someone else, please specify this in your comments.
If you work with your classmates, please make a note of it.
Any cases of academic dishonesty may 
be referred to the Office of Student Conduct.

\end{document}
