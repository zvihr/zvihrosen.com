\documentclass[12pt]{article}
\usepackage[margin=1in]{geometry}
\usepackage{pbox}
\usepackage{graphicx}
\usepackage{booktabs} % Top and bottom rules for table
\usepackage{amsfonts, amsmath, amsthm, amssymb}
\usepackage{longtable,array,color,xcolor}
\usepackage[colorlinks = true,
            urlcolor  = blue]{hyperref}

\newcommand\narrowstyle{\SetTracking{encoding=*}{-50}\lsstyle}

\pagestyle{empty}

%\usepackage[top=.6in,bottom=0.5in,left=0.5in,right=0.5in]{geometry}

\setlength{\parindent}{0pt}
%\usepackage{enumerate}
\usepackage[shortlabels]{enumitem}
\setlist{noitemsep}
%\setlist[enumerate]{topsep=0pt,itemsep=-1ex,partopsep=1ex,parsep=1ex}

\begin{document}
{\bf \large Math 320 -- Computer Methods in the Mathematical Sciences I.}\\

\vspace{3mm}

{\bf \large Project Title:} Principal Component Analysis\\
\vspace{2mm}

{\bf \large Author:} Justin Kim\\

\vspace{3mm}

\section{Presentation}
{\bf \large Comments:}
\begin{itemize}
\item Slides do a good job of giving intution.
\item You do a nice job of addressing your audience directly,
and answering people's questions.
\item Small example with data: good.
\item It would have been nice to see the PCA of the data example
visualized.
\end{itemize}

{\bf \large Grade:} 9.5/10
Great presentation -- would have benefited with some more
explicit examples and computations.


\section{Paper}

{\bf \large Comments:}
\begin{itemize}
\item Code is well-implemented with good comments.
\item Good idea to include complexity discussion!
\item Toy example is nicely implemented.
\item Comparison of PCA to OLS is nicely done -- good job including
the graphs of the two lines. Putting them on the same graph would
have made the comparison even clearer.
\item Sections 6 and 7 are interesting, but it is not clear from your 
exposition what insight the PCA adds.
\end{itemize}


{\bf \large Grade:} 9.5/10

Overall a clear and well-structured exposition of PCA. It would
have benefited from some explanation of how PCA practically aids data
analysis.
\end{document}
