\documentclass[12pt]{article}
\usepackage[margin=1in]{geometry}
\usepackage{pbox}
\usepackage{graphicx}
\usepackage{booktabs} % Top and bottom rules for table
\usepackage{amsfonts, amsmath, amsthm, amssymb}
\usepackage{longtable,array,color,xcolor}
\usepackage[colorlinks = true,
            urlcolor  = blue]{hyperref}

\newcommand\narrowstyle{\SetTracking{encoding=*}{-50}\lsstyle}

\pagestyle{empty}

%\usepackage[top=.6in,bottom=0.5in,left=0.5in,right=0.5in]{geometry}

\setlength{\parindent}{0pt}
%\usepackage{enumerate}
\usepackage[shortlabels]{enumitem}
\setlist{noitemsep}
%\setlist[enumerate]{topsep=0pt,itemsep=-1ex,partopsep=1ex,parsep=1ex}

\begin{document}
{\bf \large Math 320 -- Computer Methods in the Mathematical Sciences I.}\\

\vspace{3mm}

{\bf \large Project Title:} Forecasting Birth Rates by Race\\
\vspace{2mm}

{\bf \large Authors:} Rachel Hong \& Joshulyne Park\\

\vspace{3mm}

\section{Presentation}
{\bf \large Comments:}
\begin{itemize}
\item Good outline! Slideshow is very nicely laid out.
\item Nice description of linear regression.
\item Nice job rolling with the surprise that a slide was missing, and improvising a correction.
\item Would have been nice to see some code executing the ARIMA model.
\end{itemize}

{\bf \large Grade:} 9.5

Great presentation with appropriate depth in describing the mathematical
aspects of the models, though a bit short on the computational side.



\section{Paper}

{\bf \large Comments:}
\begin{itemize}
\item Where did your data come from? Citation needed.
\item What units is ``Birth Rate'' in? Also some graphs have the axes flipped.
\item I would have liked to see the results integrated with the analysis and computations (as I mentioned in my notes on the first draft)
\item Conclusion could have gone deeper into the dis/advantages of
the various models.
\item Nice exposition on linear regression and exponential regression!
\item The ARIMA model section is much improved -- nicely done!
\end{itemize}


{\bf \large Grade:} 8.5

There is a lot of great content here, but the current structure makes it
a bit harder to read. 



\end{document}
